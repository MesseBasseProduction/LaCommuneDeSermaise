%\documentclass[../Manuscrit.tex]{subfiles}
\documentclass[../eBook.tex]{subfiles}

\begin{document}
  \chapter*{Sources}
    \subsection*{Ouvrages}
      \paragraph{[1]}\bsc{Legrand} Martin Eugène, \textit{Commune de Sermaise, le 21 septembre 1899}, livre manuscrit, auto-édité, \bsc{Sermaise}, 1899, 36p.
      \paragraph{[2]}\bsc{Leroy} Gabriel, \textit{L'effroyable assassinat commis à Sermaise, commune de Bois-le-Roi (Seine-et-Marne) par la bande des chauffeurs, le 17 germinal an IV}, édition A.Hérisé, \bsc{Melun}, 1866, 10p.

    \subsection*{Archives}
      % Monographie de Martin Eugène Legrand
      \paragraph{[3]}\bsc{Legrand} Martin Eugène, \textit{Commune de Sermaise, le 21 septembre 1899}, monographie de l'instituteur, Archives départementales de l’Essonne, \bsc{Chamarande}, cote 1MI/98, vues 1 à 35.
      \medskip

      \hspace*{\fill}\href{https://archives.essonne.fr/ark:/28047/7bgzcx34108n}{\color{blue}/ark:/28047/7bgzcx34108n}\\
      \hspace*{\fill}\textit{Consulté le 25 mars 2022}
      % Naissance de Martin Eugène Legrand à Isbergues
      \paragraph{[4]}\textit{Registres paroissiaux et d'état civil de la commune d'Isbergues}, registre BMSN entre 1598 et 1885, Archives départementales du Pas-de-Calais, \bsc{Arras}, cote 5MIR473/2, vue 1394, acte n°9.
      \medskip

      \hspace*{\fill}\href{https://archivesenligne.pasdecalais.fr/v2/ark:/64297/d48a62292abef7f5e365db8b88b6d37d}{\color{blue}/ark:/64297/d48a62292abef7f5e365db8b88b6d37d}\\
      \hspace*{\fill}\textit{Consulté le 29 janvier 2025}
      % Mariage de Martin Eugène Legrand et de Marie Adélaïde Fessenet à Corbeil-Essonnes
      \paragraph{[5]}\textit{Registre d'état civil de la commune de Corbeil-Essonnes}, registre M en 1891, Archives départementales de l’Essonne, \bsc{Chama-\\rande}, cote 4E/3210, vue 131, acte n°81.
      \medskip

      \hspace*{\fill}\href{https://archives.essonne.fr/ark:/28047/h7bl0sk48qg5/3d3ec9db-364d-45d5-8f27-a290fdeb022b}{\color{blue}/ark:/28047/h7bl0sk48qg5}\\
      \hspace*{\fill}\textit{Consulté le 4 février 2025}
      % Domicilation de Martin Eugène Legrand et de Marie Adélaïde Fessenet à Sermaise
      \paragraph{[6]}\textit{Liste nominative des habitants de la commune de Sermaise}, recensement de la population de 1901, Archives départementales de l’Essonne, \bsc{Chamarande}, cote 6M/279, vue 3, ménage n°10.
      \medskip

      \hspace*{\fill}\href{https://archives.essonne.fr/ark:/28047/dmx3pw9lj7gc/7a92f1bf-1fd2-4549-8d7f-c9b250262e85}{\color{blue}/ark:/28047/dmx3pw9lj7gc}\\
      \hspace*{\fill}\textit{Consulté le 29 janvier 2025}
      % Naissance de Louis Eugène Legrand à Sermaise
      \paragraph{[7]}\textit{Registre d'état civil de la commune de Sermaise}, registre NMD entre 1889 et 1905, Archives départementales de l’Essonne, \bsc{Chama-\\rande}, cote 4E/3828, vue 213, acte n°16.
      \medskip

      \hspace*{\fill}\href{https://archives.essonne.fr/ark:/28047/vs1wd9xnj6lq/b79e3835-828c-4169-a897-6eb3cff121d2}{\color{blue}/ark:/28047/vs1wd9xnj6lq}\\
      \hspace*{\fill}\textit{Consulté le 30 janvier 2025}
      % Décès de Marie Adélaïde Fessenet à Massy
      \paragraph{[8]}\textit{Registre d'état civil de la commune de Massy}, registre NMD en 1908, Archives départementales de l’Essonne, \bsc{Chamarande}, cote 4E/4833, vue 3, acte n°5.
      \medskip

      \hspace*{\fill}\href{https://archives.essonne.fr/ark:/28047/gpbfh50cxk2j/47c982d3-98aa-4660-a8ec-319d75a6cebd}{\color{blue}/ark:/28047/gpbfh50cxk2j}\\
      \hspace*{\fill}\textit{Consulté le 30 janvier 2025}
      % Re-mariage de Martin Eugène Legrand et de Marcelle Charles Léonie Villion à Mennecy
      \paragraph{[9]}\textit{Registre d'état civil de la commune de Mennecy}, registre NMD entre 1905 et 1912, Archives départementales de l’Essonne, \bsc{Chama-\\rande}, cote 4E/4852, vue 330, acte n°52.
      \medskip

      \hspace*{\fill}\href{https://archives.essonne.fr/ark:/28047/xdkqf0j4h825/ffc7092b-bcb4-4189-bb4a-037494b9199c}{\color{blue}/ark:/28047/xdkqf0j4h825}\\
      \hspace*{\fill}\textit{Consulté le 4 février 2025}
      % Sermaise environnement
      \paragraph{[10]}\textit{La Commune de Sermaise 1899 par E Legrand}, transcription partielle de la monographie sur une page web, Sermaise Environnement, \bsc{Sermaise}.
      \medskip

      \hspace*{\fill}\href{https://web.archive.org/web/20210923214057/https://www.sermaise-environnement.org/communedesermais/index.html}{\color{blue}https://sermaise-environnement.org}\\
      \hspace*{\fill}\textit{Consulté le 23 septembre 2021}
    \newpage
\end{document}
