%\documentclass[../Manuscrit.tex]{subfiles}
\documentclass[../eBook.tex]{subfiles}

\begin{document}
  \chapter*{Préface}
    \paragraph{}À la fin du XIX\up{e} siècle, à l'aube de l'Exposition universelle de 1900 s'étant tenue à Paris, la France s'est livrée à un exercice de mémoire et d'inventaire d'un genre singulier. Le 29 décembre 1898, le ministère de l'Instruction publique demanda à tous les instituteurs du pays de rédiger une monographie communale et scolaire.
    \paragraph{}Il ne s'agissait pas simplement d'un relevé administratif, mais d'un véritable portrait vivant de chaque commune, tel que vu, vécu et décrit par celui ou celle qui enseignait au cœur de son village.
    \paragraph{}La monographie est une étude descriptive, souvent minutieuse, consacrée à un sujet unique : dans notre cas présent, les communes françaises. Ce genre, à la croisée de l'analyse, de l'enquête et du témoignage, offre un portrait complet du territoire : son histoire, sa géographie, ses activités économiques, sa population, ses us et coutumes, et surtout son école, reflet de la mission éducative de la République.
    \paragraph{}L'administration fixa un cadre rédactionnel aux instituteurs, afin que chaque monographie, bien que singulière par son contenu, suive une trame commune décrivant l'ensemble des aspects évoqués plus haut.
    \paragraph{}C'est au sein des Archives départementales de l'Essonne que la monographie de la commune de Sermaise, rédigée en 1899, est précieusement conservée. Elle fait partie d'un corpus exceptionnel de 184 monographies de communes pour le seul département de l'Essonne, anciennement Seine-et-Oise.
    \paragraph{}Cette mise à disposition publique a motivé le désir d'en faire une transcription intégrale, afin d'en rendre le contenu accessible à tous, sans les barrières du manuscrit ou de l'écriture d'époque qui, bien que resplendissante, requiert une certaine adaptation à la lecture.
    \paragraph{}Le village de Sermaise a eu la chance d'être ainsi raconté par son instituteur de l'époque ; Martin Eugène Legrand. Né en 1865 à Isbergues, dans le Pas-de-Calais, il consacra sa vie à l'enseignement et à la transmission du savoir au fil d'une carrière qui l'amena à Sermaise, à Massy puis à Corbeil-Essonnes.
    \paragraph{}Marié à Marie Adélaïde Fessenet, ils eurent un fils né à Sermaise en 1902, Louis Eugène qui à son tour devint instituteur. Martin Eugène Legrand décéda en 1948 à Corbeil, après une vie tout entière dédiée à l'éducation publique. Son regard d'instituteur -- à la fois rigoureux, sensible et ancré dans les réalités du quotidien -- donne à cette monographie une richesse documentaire et humaine rare.
    \paragraph{}La transcription ici présentée se veut fidèle au texte original de Martin Eugène Legrand : aucun mot, aucune tournure ni formulation n'ont été modifiées. La langue est restée celle de 1899, avec son charme, sa structure, ses archaïsmes, son orthographe, sa typographie -- autant de marques précieuses de son temps.  Par souci de lisibilité en revanche, et ce afin d'en faciliter la lecture contemporaine, certains des plus longs paragraphes ont été scindés.
    \paragraph{}Ces interventions ne modifient en rien le fond, mais permettent une meilleure appréhension de la richesse du texte. Enfin, des notes de bas de page signalées par la mention \textit{NDLR} (Notes du lecteur--rédacteur) viennent éclairer certains termes ou passages aujourd'hui moins accessibles, ou bien souligner des fautes présentes dans le manuscrit original. Ces éclaircissements ponctuels n'ont d'autre but que d'ouvrir davantage encore le texte à tous les publics, sans jamais trahir la voix de l'auteur. Les autres notes de bas de pages sont, en revanche, issues du manuscrit original.
    \paragraph{}C'est grâce au travail de classement, de conservation et de valorisation des Archives départementales de l'Essonne que cette monographie a pu ressurgir du silence des allées d'archives. Sans leur engagement rigoureux à préserver ce patrimoine discret mais fondamental, ce document serait probablement resté inconnu, perdu dans l'anonymat des liasses administratives.
    \paragraph{}Leur action ne se limite pas à l'entretien d'un héritage ; elle en permet la transmission vivante, en rendant ces textes accessibles à tous les citoyens désireux de mieux comprendre le passé de leur territoire.
    \paragraph{}C'est dans ce contexte que l'idée de retranscrire et de publier la monographie de Sermaise a pu naître, nourrie par la volonté de donner une seconde vie à un témoignage marquant, et de partager avec le plus grand nombre cette voix du XIX\up{e} siècle, restée intacte grâce à la mémoire patiemment entretenue par les archivistes.
    \paragraph{}Ce modeste volume n'est pas seulement un document administratif ancien. Il est le témoignage rare d'un village vu de l'intérieur, raconté par un homme attentif, rigoureux et profondément attaché à sa mission d'enseignement. Il est à la fois un témoignage d'un temps révolu, un hommage au travail silencieux des enseignants d'antan, et un outil de redécouverte pour les habitants d'aujourd'hui. Puisse-t-il raviver la mémoire locale, inspirer la curiosité historique, et rappeler, à sa manière, combien l'éducation fut -- et reste -- un pilier essentiel de notre société.
\end{document}
