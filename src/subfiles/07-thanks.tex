%\documentclass[../Manuscrit.tex]{subfiles}
\documentclass[../eBook.tex]{subfiles}

\begin{document}
  \paragraph{}Je tiens à exprimer ma profonde gratitude à mes correctrices, Aliénor \bsc{Beaulieu}, Anne \bsc{Chevy} et Pauline \bsc{Dufermon} pour leur œil attentif et bienveillant.
  \paragraph{}Merci également aux Archives départementales de l'Essonne qui m'ont permis de consulter le manuscrit original, rédigé de la main de Legrand. Je remercie particulièrement Amandine \bsc{Metraux} pour son accueil et ses conseils en salle de lecture.
  \paragraph{}Ma reconnaissance va aussi à l'association Sermaise Environnement qui, grâce à un article sur leur site internet, a éveillé mon intérêt pour cette archive exceptionnelle ; c'est sans nul doute ce qui m'a conduit aux Archives départementales et, \textit{in fine}, à la transcription de cette monographie.
  \paragraph{}Enfin, j'adresse un merci tout spécial à ma femme, Pauline, pour sa patience, son indéfectible soutien et cette bienveillance avec laquelle elle m’accompagne, pas à pas, dans le tumulte de mes mille projets.
  \begin{flushright}
    \textit{Arthur}
  \end{flushright}
  \vspace*{\fill}
  \begin{center}
    \begin{tabular}{r l}
      \textbf{Auteur} & Martin Eugène \bsc{Legrand}\\
      \textbf{Transcription} & Arthur \bsc{Beaulieu}\\
      &\\
      \textbf{Édition} & \bsc{Messe Basse Production}\\
      &\\
      \textbf{Relecture} & Aliénor \bsc{Beaulieu}\\
      & Anne \bsc{Chevy}\\
      & Pauline \bsc{Dufermon}\\
      &\\
      \textbf{Maquette} & Arthur \bsc{Beaulieu}
    \end{tabular}
  \end{center}
  \vspace{-24pt}
  \normalsize{}
  \newpage
\end{document}
